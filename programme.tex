% Programme for performance in Trinity Hall, 2025

\documentclass[10pt,british]{article}

\usepackage[utf8x]{inputenc}
\usepackage[a5paper,noheadfoot]{geometry}
\usepackage[english]{babel}
\usepackage{newtxtext,tabularx,url}

\newcommand{\mytitle}[1]%
  {\centerline{\normalfont\large\bfseries#1\\*[\bigskipamount]}}
\newcommand{\mysection}[1]%
  {\centerline{\normalfont\large\bfseries#1}\bigskip}


\begin{document}
\pagestyle{empty}
\begin{bfseries}
\vspace*{2cm}\bigskip\bigskip
\centerline{\huge \emph{Leaf by Niggle}}
\bigskip\bigskip
\centerline{\LARGE by}
\bigskip\bigskip
\centerline{\huge J. R. R. Tolkien}
\bigskip\bigskip
\end{bfseries}
\vspace{3cm}
{\Large\bfseries
\begin{center}
\renewcommand{\arraystretch}{1.5}
\begin{tabularx}{0.8\linewidth}{Xl}
  Laurence Keegan-Fischer & \emph{Piano} \\
  Reuben Thomas & \emph{Reader} \\
\end{tabularx}
\end{center}
}
\vspace{1cm}
\begin{center}
\bfseries
Wong--Avery Room \\ Trinity Hall \\ Thursday 30th October 2025 \\
\end{center}
\pagebreak

\mysection{Performers}

\noindent \textbf{Reuben Thomas}, a singer, software developer and lyricist, first read \emph{Leaf by Niggle} as a child. In the early 2000s he started to read it aloud to friends, along with stories by G.\,K.~Chesterton, Antoine de St-Exupéry and others. Since 2017, Reuben has appeared in many of emeritus Professor Patrick Boyde’s semi-staged dramas, including as Sam(p)son in a double bill of the Greek Septuagint story of Sampson and Milton’s \emph{Samson Agonistes}, as the doomed lover in Tennyson’s \emph{Maud}, and the emperor Titus in Racine’s \emph{Bérénice}. In COVID lockdown, Reuben recorded Alexander Pope’s versions of the \emph{Iliad} and \emph{Odyssey}. See https://boydesclassicdramas.org for these and many more, and https://rrt.sc3d.org/Readings for a selection of short stories and poetry read by Reuben.

\bigskip

\noindent \textbf{Laurence Keegan-Fischer} tunes pianos.

\vspace{1cm}
\mysection{Programme}

\noindent Tolkien wrote \emph{Leaf by Niggle} in 1939, after waking up from a dream with the story complete in his mind. It was first published in 1945. Unlike his familiar Middle Earth mythos, \emph{Niggle} is a modern allegory whose form closely parallels Dante’s \emph{Purgatorio}.

Reuben has wanted to perform the story with music for over 30 years, but it was only after he met Laurence in 2017 that he started to work out how. Together, they developed the idea of spoken words accompanied by Anglican chant. Laurence adapted the chants from the music of Ralph Vaughan Williams, which both chronologically and stylistically seemed a good match for Tolkien.

\end{document}

%%% Local Variables:
%%% mode: latex
%%% TeX-master: t
%%% End:
